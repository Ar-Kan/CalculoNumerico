%Este trabalho está licenciado sob a Licença Creative Commons Atribuição-CompartilhaIgual 3.0 Não Adaptada. Para ver uma cópia desta licença, visite https://creativecommons.org/licenses/by-sa/3.0/ ou envie uma carta para Creative Commons, PO Box 1866, Mountain View, CA 94042, USA.

\chapter*{Nota dos organizadores}
\addcontentsline{toc}{chapter}{Nota dos organizadores}

Nosso objetivo é de fomentar o desenvolvimento de materiais didáticos pela colaboração entre professores e alunos de universidades, institutos de educação e demais interessados no estudo e aplicação de cálculo numérico nos mais diversos ramos da ciência e tecnologia.

Para tanto, disponibilizamos em repositório público GitHub todo o código-fonte dos materiais em desenvolvimento sob licença Creative Commons Atribuição-CompartilhaIgual 3.0 Não Adaptada (\href{https://creativecommons.org/licenses/by-sa/3.0/}{CC-BY-SA-3.0}). Ou seja, você pode copiar, redistribuir, alterar e construir um novo material para qualquer uso, inclusive comercial. Leia a licença para maiores informações.

O sucesso do projeto depende da colaboração! Participe diretamenta da escrita dos recursos educacionais, dê sugestões ou nos avise de erros e imprecisões. Toda a colaboração é bem-vinda. Veja mais sobre o projeto em:
\begin{center}
  \url{https://www.ufrgs.br/reamat/CalculoNumerico}
\end{center}

% Estamos escrevendo este livro de forma colaborativa desde 2011 e, recentemente, decidimos por abrir a colaborações externas. Nosso objetivo é produzir um material didático no nível de graduação de excelente qualidade e de acesso livre pela colaboração entre professores e alunos de universidades, institutos de educação e demais interessados na análise, estudo e aplicação de métodos numéricos nos mais diversos ramos da ciência e da tecnologia.

% O sucesso do projeto depende da colaboração! Edite você mesmo o livro, dê sugestões ou nos avise de erros e imprecisões. Toda a colaboração é bem-vinda. Saiba mais visitando o site oficial do projeto:
% \begin{center}
%   \url{https://www.ufrgs.br/reamat}
% \end{center}

% Nós preparamos uma série de ações para ajudá-lo a participar. Em primeiro lugar, o acesso irrestrito ao livro pode se dar através do \href{https://www.ufrgs.br/numerico}{site oficial do projeto}.
% %%%%%%%%%%%%%%%%%%%%
% % scilab
% %%%%%%%%%%%%%%%%%%%%
% \ifisscilab
% Disponibilizamos o livro na versão original em \href{https://www.ufrgs.br/numerico/livro/main.pdf}{PDF} e versões adaptadas em \href{https://www.ufrgs.br/numerico/livro/main.html}{HTML}, \href{https://www.ufrgs.br/numerico/livro/main.epub}{EPUB} e \href{https://www.ufrgs.br/numerico/livro/slide.pdf}{Slides}.
% \fi
% %%%%%%%%%%%%%%%%%%%%
% %%%%%%%%%%%%%%%%%%%%
% % octave
% %%%%%%%%%%%%%%%%%%%%
% \ifisoctave
% Disponibilizamos o livro na versão original em \href{https://www.ufrgs.br/numerico/livro-oct/main-oct.pdf}{PDF} e versões adaptadas em \href{https://www.ufrgs.br/numerico/livro-oct/main.html}{HTML}, \href{https://www.ufrgs.br/numerico/livro-oct/main-oct.epub}{EPUB} e \href{https://www.ufrgs.br/numerico/livro-oct/slide-oct.pdf}{Slides}.
% \fi
% %%%%%%%%%%%%%%%%%%%%
% %%%%%%%%%%%%%%%%%%%%
% % python
% %%%%%%%%%%%%%%%%%%%%
% \ifispython
% Disponibilizamos o livro na versão original em \href{https://www.ufrgs.br/numerico/livro-py/main-py.pdf}{PDF} e versões adaptadas em \href{https://www.ufrgs.br/numerico/livro-py/main.html}{HTML}, \href{https://www.ufrgs.br/numerico/livro-py/main-py.epub}{EPUB} e \href{https://www.ufrgs.br/numerico/livro-py/slide-py.pdf}{Slides}.
% \fi
% %%%%%%%%%%%%%%%%%%%%
% Além disso, o livro está escrito em código \LaTeX{} disponível em \href{https://github.com/livroscolaborativos/CalculoNumerico}{repositório GitHub público}.

% Nada disso estaria completo sem uma licença apropriada à colaboração. Por isso, escolhemos disponibilizar o material do livro sob licença \href{https://creativecommons.org/licenses/by-sa/3.0/}{Creative Commons Atribuição-CompartilhaIgual 3.0 Não Adaptada (CC-BY-SA 3.0)}. Ou seja, você pode copiar, redistribuir, alterar e construir um novo material para qualquer uso, inclusive comercial. Leia a \href{https://creativecommons.org/licenses/by-sa/3.0/}{licença} para maiores informações.

\vspace{0.5cm}

Desejamos-lhe ótimas colaborações!
